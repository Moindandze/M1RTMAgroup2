\chapter{User requirements}% * c est pour ne pas numerote le chapitre
\addcontentsline{toc}{chapter}{User requirements}


\section{Project environment}

\begin{enumerate}
	\item Context and environment \paragraph{}
Previously carried out at the SIC Department of XLIM laboratory, the project we are in charge of implements "algorithmic tools and capturing software for facial recognition". This project interest was also interactions with and between users of a video game with an educational aim ("Serious game") by automatically animated avatars and experienced analysis difficulties, as well as the development of a software library for developers of video games on smartphones.\paragraph{}
Facial recognition is often used as a tool to secure devices or control the use of applications. This technique both helps adults secure their systems and devices and parents limit and control access to systems for their kids.\paragraph{}
The main idea was to improve parental control on the use of tablets for children. This could be helpful to parents wishing to use facial recognition to limit their children's use past a certain time of a day.
\vspace{0.5cm}
\end{enumerate}

\begin{enumerate}
	\item Project Objective \paragraph{}
The work required for this project is the development of a share recording software for the recognition of facial expressions for the identification of a face from an ID list in a home or for parental control.\paragraph{}
Today innovation and the latest technologies are growing increasingly and allow the interactivity between human-machine to be maximal. Research facial expression is fundamental in many applications.\paragraph{}
Facial recognition takes place in three stages, namely:
\begin{itemize}
	\item Face detection;
	\item Extraction and normalization of facial features;
	\item Identification and / or verification.\paragraph{}
\end{itemize}    

The main difficulty in face recognition is that no two identical faces. Thus, each individual is unique and will be marked by gender, ethnicity, age or his haircut, but also by the shape, size and arrangement of the elements of the face.\paragraph{}

This project has an educational goal because it contributes greatly to our engineering multimedia training, allowing us to put into practice the theories studied in the various teaching modules of our master degrees (Image processing, tool and scientific computation, Algorithms for multimedia, random signal processing) but also by completing them. This project can be seen as a complement to art training.\vspace{0.5cm}

\end{enumerate}

\begin{enumerate}
	\item Needs analysis instead Statement of requirements\paragraph{}
This project aims to develop a facial capture software program (face recognition) in XLIM-SIC laboratory research work in collaboration with a company based in Lyon operating in the field of manufacturing tablets for children.\paragraph{}
The required results at the end of this project are: facial recognition software program with:
\begin{itemize}
	\item Eigen Faces;
	\item Fisher Faces.\paragraph{}
\end{itemize}

Those will be programmed with multiple programming languages (Python, C ++).\paragraph{}
This project is performed with the research professors of the Fondamental Faculty of  Sciences at the University of Poitiers: Pascal Bourdon and David HELBERT that supervises.\vspace{0.5cm}

\end{enumerate}

\begin{enumerate}
	\item Actors\paragraph{} 
The sponsor of this project is M. Pascal Bourdon working for XLIM-SIC Laboratory in Poitiers as a researcher. The results will be used by the Laboratory.\paragraph{}
The project team consists of:
Supervisors:
\begin{center}
Pascal Bourdon supervises;\\
David HELBERT teacher researcher Labo XLIM - SIC.\paragraph{}
\end{center}

Students in charge:
\begin{center}
Viviane Arame BASSE as Communication Manager;\\
Guy Florent A. SADELER as Technical Manager;\\
Ali TOILHA as Project Manager.\paragraph{}
\end{center}

The end-users of this project are:
\begin{center}
XLIM-SIC Laboratory;\\
RTMA Training of the Poitiers University for tutorials;\\
Potential client company in need of facial recognition technology.\vspace{0.5cm}
\end{center}

\end{enumerate}

\begin{enumerate}
	\item Description of the existing \paragraph{}

We have for Completion of this project :
\begin{itemize}
	\item A database images for face recognition,
	\item Documents on the documentation of Open CV. \vspace{0.5cm}
\end{itemize}

\end{enumerate}


\section{Constraints}

\begin{enumerate}
	\item	Time\paragraph{}
	
	\begin{itemize}
		\item Delivery date of product / service is scheduled:  June 10, 2015 at 4 pm.
		\item Intermediate dates:
		\begin{flush-left}
			April 2: Users requirements presentation + handouts;\\
			May 13: State of the art Theoretical;\\
			May 27: Intermediate technical report;\\
			June 10: Official delivery (technical documents, codes, ...);\\
			June 17: Project presentation.\vspace*{0.5cm}
		\end{flush-left}
	\end{itemize}

\end{enumerate}



\section{ Product Description or final service}


At the end of the project we are asked to deliver a deliverable consists of :
\begin{center}
A specification,\\
A technical report,\\
The codes programmed with their documentation. \vspace*{0.5cm}
\end{center}



\section{Sequence and Organization of the project }

	\begin{enumerate}
		\item Planning \paragraph{}
		
		\begin{itemize}
			\item Documentation and Bibliography;
			\item Drafting of the specifications;
			\item Preparation of the presentation of the specifications;
			\item Drafting of the state of the art;
			\item Intermediate Report (codes if possible);
			\item Prototyping eigenfaces in python;
			\item Prototyping Fisherfaces in python;
			\item Configuration Management;
			\item Writing code documentation and technical report;
			\item Unit Testing and validation;
			\item Comparison of the results related to the methods used;
			\item Delivery of the final report.\vspace*{0.5cm}
		\end{itemize}
		
		\item Location \paragraph{}
The project will take place in the project room EEA of building SP2MI.\vspace*{0.5cm}
	\end{enumerate}











